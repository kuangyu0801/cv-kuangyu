%%%%%%%%%%%%%%%%%%%%%%%%%%%%%%%%%%%%%%%%%
% Medium Length Professional CV
% LaTeX Template
% Version 2.0 (8/5/13)
%
% This template has been downloaded from:
% http://www.LaTeXTemplates.com
%
% Original author:
% Trey Hunner (http://www.treyhunner.com/)
%
% Important note:
% This template requires the resume.cls file to be in the same directory as the
% .tex file. The resume.cls file provides the resume style used for structuring the
% document.
%
%%%%%%%%%%%%%%%%%%%%%%%%%%%%%%%%%%%%%%%%%

%----------------------------------------------------------------------------------------
%	PACKAGES AND OTHER DOCUMENT CONFIGURATIONS
%----------------------------------------------------------------------------------------

\documentclass{resume} % Use the custom resume.cls style
\usepackage{xcolor} % includes color package
\usepackage{hyperref}

\usepackage[left=0.5in,top=0.5in,right=0.5in,bottom=0.25in]{geometry} % Document margins

\name{Kuang-Yu Li} % Your name
\headline{IT Master Student  focus on Cloud Computing} % Your professional headline
\address{Allmandring 20D, Stuttgart, Germany 70569 \\ +49 152 07439908 } % Your address
\address{\href{mailto:kuangyu.li@outlook.com}{kuangyu.li@outlook.com} \\ \href{https://www.linkedin.com/in/kuang-yu-li-lumiere/}{LinkedIn/kuang-yu-li-lumiere} \\ \href{https://github.com/kuangyu0801}{GitHub/kuangyu0801}} % Your phone number and email
\begin{document}


%----------------------------------------------------------------------------------------
%	EDUCATION SECTION
%----------------------------------------------------------------------------------------

\begin{rSection}{Education}
{\bf Universität Stuttgart }{--  Stuttgart, Germany}  \hfill {Oct. 2019 - present}\\
{\em M.S. in Information Technology}{, German Grading: 1.8 (Gut) \\ 
{\bf National Chiao Tung University (NCTU) }{--  Hsinchu, Taiwan} \hfill {Sept. 2009 - Oct. 2015}\\
{\em B.S. in Electrical Engineering and Computer Science}{, GPA: 3.9/4.3}
\em M.S. in Electronics Engineering and Electronics}{, GPA: 4.27/4.3} 

\end{rSection}

%----------------------------------------------------------------------------------------
%	WORK EXPERIENCE SECTION
%----------------------------------------------------------------------------------------
\begin{rSection}{Experience}

\begin{rSubsection}{IBM}{--  Böblingen, Germany}{Apr. 2021 - present}{Intern, IBM Systems}
\item Developing a software as IDE plug-in with remote development capabilities on IBM Z mainframe servers including debugging, static code analysis and unit test integration. 
\end{rSubsection}

\begin{rSubsection}{MediaTek, Inc.}{--  Hsinchu, Taiwan}{Dec. 2015 - Aug. 2019}{Firmware Engineer, Communication System Design}

\item Developed embedded firmware in MediaTek’s Android platform for 4G LTE and 5G NR physical layer (PHY) digital signal processing
\item Contributed to 6k and maintained 20k lines of code in C/C++ for 3 large-scale projects (over 200k lines and 1,000 developers) including 2019 world’s fastest 5G modem  \href{https://www.mediatek.com/blog/heres-5gs-real-speed-live-connection-demo
}{Helio M70} with downlink throughput of 4.7 Gbps
\item Designed mobile modem receiver modules, implemented OFDM signal processing algorithms accordant with  \href{https://www.3gpp.org/specifications}{3GPP Spec}, integrated build and auto testing process for design and system verification with Perl and Python scripts 
\item Resolved urgent system function and performance issues for over 10 MediaTek’s \href{https://www.mediatek.com/products/smartphones}{Helio X and P series smartphone products} with log examination, signal and procedure analysis, solution formulation and implementation, patch verification and release

\item Received 7 MediaTek vAwards (top 10\% monthly team performance) including  improving 4G downlink data rate by 300\% (14$\rightarrow$42 Mbps) with Massive MIMO feature and reducing 5G control channel decoding computation  cycles by 72\% (3700$\rightarrow$1000 cycles)

\end{rSubsection}

\end{rSection}

%----------------------------------------------------------------------------------------
%	TECHNICAL STRENGTHS SECTION
%----------------------------------------------------------------------------------------

\begin{rSection}{Technical Strengths}
\begin{tabular}{ @{} >{ \bfseries}l @{ \hspace{1ex}} l}

% Field & Software Development, Mobile Communication \\

{Language} & {Best at Java; Fluent in C; Familiar with C++, Python, JSP, shell scripts {\bf Tool} Git/GitHub, Docker, Apache Tomcat} \\
{Platform} & {AWS: EC2, SQS, Google Cloud: Kubernetes Engine, Functions, Pub/Sub, IBM Cloud: Hyper Protect Virtual Server}  \\
{Database} & {SQL: DynamoDB; NoSQL: Firebase, Datastore {\bf Web Technology} REST, HTTP, XML, SOAP, WSDL, HTML/CSS}  \\
\end{tabular}

%\begin{tabular}{ @{} >{\bfseries}l @{\hspace{6ex}} l }
% Computer Languages & Prolog, Haskell, AWK, Erlang, Scheme, ML \\
% Protocols \& APIs & XML, JSON, SOAP, REST \\
% Databases & MySQL, PostgreSQL, Microsoft SQL \\
% Tools & SVN, Vim, Emacs
%\end{tabular}

\end{rSection}
%----------------------------------------------------------------------------------------
%	PROJECT
%----------------------------------------------------------------------------------------

\begin{rSection}{PROJECTS}
\begin{rSubsection}{\href{https://github.com/kuangyu0801/software-defined-networking-ws20/tree/main/sdn-assign-04}{Content-Based Router for Publish-Subscribe Service}} { Java, REST API, HTTP}{Distributed System Lab}{}
\item Developed a routing controller in a Java module, which provides content-based routing for pub/sub service to minimize network traffic and reduce application filtering effort. The routing algorithm is based on sorting and merging interval of encoded IP-address 
\item Developed a "Subscriber" Java application which can subscribe to publishing service via REST API and receive UDP  from publisher.
\item Developed a one-to-many publishing service in Floodlight controller, which can receive subscription via HTTP request (GET, POST, and DELETE) and perform content-based routing in a OpenFlow network.
\end{rSubsection}

\begin{rSubsection}{\href{https://github.com/kuangyu0801/software-defined-networking-ws20/tree/main/sdn-assign-03}{Dynamic Routing for Software-Defined Network}} {Java, Dijkstra's algorithm}{Distributed System Lab}{}
\item Developed a Java module in Floodlight controller, which provides 2 dynamic routing modes in OpenFlow network. 
\item Reactive mode routes with shortest path. Adaptive mode routes TCP flow with load balancing by querying network traffic statistics dynamically and matching IP addresses and TCP ports. The implementation is based on Dijkstra's algorithm.
\item Verified application with Iperf in MiniNet on Linux and achieves 6x bandwidth increase (582kbs vs 3478kbs) in adaptive mode
\end{rSubsection}

\begin{rSubsection}{\href{https://github.com/kuangyu0801/MobileComputing_SS20_assign04}{Java Application for Wireless Ad-hoc Network}} {Java, UDP, Raspberry Pi}{Mobile Computing}{}
\item Developed 4 Java server and client applications which implement 2 protocols: Flooding and Dynamic Source Routing (DSR). Flooding achieves high robustness with UDP messages broadcast. DSR achieves reduced data transfer overhead with route discovery in control messages. Applications use DatagramSocket classes from java.net package for UDP transmission.
\item Verified applications on Raspberry Pi with real mesh 802.11 WiFi network. 

\end{rSubsection}
\begin{rSubsection}{\href{https://github.com/kuangyu0801/MobileComputing_SS20_assign03}{Android App for City Temperature with Google Firebase}}{Java, Android}{Mobile Computing}{}
\item Developed an Android application, which can update, subscribe, and calculate daily average of designated city temperature
\item Implemented functions for accessing and querying  data in JSON in a shared Realtime NoSQL database  with Google Firebase API
\end{rSubsection}
\begin{rSubsection}{\href{https://github.com/kuangyu0801/WS19_ComplexNetworkSystem}{ATP Tennis Player Network Analysis}}{Python, Graph, NetworkX}{Complex Network System}{}
\item Developed Python programs to generate complex network and derive structural insights such as Page Rank, Connectivity, Clustering, etc.
\item Implemented algorithms with NetworkX package and built an undirected graph by processing real tennis match statics in csv format.
\item Discovered, visualized, rendered and exported network topology with open-source software Gephi
\end{rSubsection}

\begin{rSubsection}{\bf \href{https://1drv.ms/p/s!AiukWIzY5GZp22FjndnKscnEaL85?e=F2UJEL}{Forest Cover Type Prediction}}{-- Python}{Machine Learning}{}
\item Implemented {\em Decision Tree} and {\em Support Vector Machine} with {\em Scikits-Learn} package, evaluated and discussed the performance on forest type classification problem 

\end{rSubsection}

\end{rSection}

%%
%\begin{rSection}{HONOR}
%{\bf NCTU Excellent Exchange Student Scholarship}  \hfill {Aug. 2012}\\
%{9,000 USD for exchange to Katholieke Universiteit Leuven, Belgium (top 2\% in NCTU)
%\end{rSection}

%%
%----------------------------------------------------------------------------------------

%----------------------------------------------------------------------------------------
%	EXAMPLE SECTION
%----------------------------------------------------------------------------------------

%\begin{rSection}{Section Name}
% {\bf #Title} \hfill {\em #Time} \\ 
% #Description

%Section content\ldots

%\end{rSection}

%----------------------------------------------------------------------------------------

\end{document}
