%%%%%%%%%%%%%%%%%%%%%%%%%%%%%%%%%%%%%%%%%
% Medium Length Professional CV
% LaTeX Template
% Version 2.0 (8/5/13)
%
% This template has been downloaded from:
% http://www.LaTeXTemplates.com
%
% Original author:
% Trey Hunner (http://www.treyhunner.com/)
%
% Important note:
% This template requires the resume.cls file to be in the same directory as the
% .tex file. The resume.cls file provides the resume style used for structuring the
% document.
%
%%%%%%%%%%%%%%%%%%%%%%%%%%%%%%%%%%%%%%%%%

%----------------------------------------------------------------------------------------
%	PACKAGES AND OTHER DOCUMENT CONFIGURATIONS
%----------------------------------------------------------------------------------------

\documentclass{resume} % Use the custom resume.cls style
\usepackage{xcolor} % includes color package
\usepackage{hyperref}

\usepackage[left=0.5in,top=0.5in,right=0.5in,bottom=0.2in]{geometry} % Document margins

\name{Kuang-Yu Li} % Your name
\headline{Full Stack  Developer, Cloud Specialist, DevOps Advocate} % Your professional headline
\address{+49 152 07439908 \\ \href{mailto:kuangyu.li@outlook.com}{kuangyu.li@outlook.com} \\ \href{https://www.linkedin.com/in/kuang-yu-li-lumiere/}{LinkedIn/kuang-yu-li-lumiere} \\ \href{https://github.com/kuangyu0801}{GitHub/kuangyu0801}} % Your phone number and email
\begin{document}

%----------------------------------------------------------------------------------------
%	WORK EXPERIENCE SECTION
%----------------------------------------------------------------------------------------
\begin{rSection}{Experience}

\begin{rSubsection}{Google}{-- Taipei, Taiwan}{Mar. 2022, - present (9 month)}{Software Engineer, Device and Services, Developing Modem System Software on Pixel phones and wearables}
\item Developed Google Embedded Modem System (GeMS) as an RTOS framework to provide high quality and reusability modem system code for Pixel phones and wearables based on \href{https://pigweed.dev/}{Pigweed}, a open source  embedded-targeted libraries. (C++17, GN Build, Clang compiler)
\item  Designed and Implemented 2 GeMS OS primitives: Memory and Queue for \href{https://semiconductor.samsung.com/processor/modem/exynos-modem-3500/}{Exynos Modem 3500} and \href{https://www.qualcomm.com/products/application/wearables/snapdragon-w5-plus-gen-1-wearable-platform} {Snapdragon SW5100} platform and implemented 3 Pigweed OS primitive backends : \href{https://pigweed.dev/pw_chrono/#software-timers}{System Timer}, \href{https://pigweed.dev/pw_sync/#threadnotification}{Thread Notification} and \href{https://pigweed.dev/pw_assert/#pw-assert-api-backend}{Assertion Log} on ShannonOS.
\item Facilitated GeMS infra to follow Google SW core practices by 1) configuring static analysis(clang-tidy), formatting(clang-format), and license check(addlicenses) in presubmit checks to improve code quality and readability and 2) developing a Plugweed plugin for automatic API reference Doc generation (Doxygen)

% https://blog.google/products/devices-services/	
\end{rSubsection}

\begin{rSubsection}{IBM}{--  Böblingen, Germany}{Apr. 2021 - Sept. 2021 }{Intern, IBM Systems: Modernized mainframe development environment}
\item Developed a TypeScript cloud-native remote IDE for \href{https://www.ibm.com/it-infrastructure/z/os/linux}{Ubuntu on IBM Z15}  based on \href{https://theia-ide.org/}{Eclipse Theia} and \href{https://code.visualstudio.com/api}{VSCode Extension API} using Scrum
\item Verified IDE feasibility and compatibility by deploying and testing  \href{https://theia-ide.org/}{Eclipse Theia} on Hyper Protect Server and \href{https://www.eclipse.org/che/} {Eclipse Che} on Minikube
\item Integrated, tested, and deployed C++ dev tools to  \href{https://www.redhat.com/en/technologies/jboss-middleware/codeready-workspaces}{Codeready Workspace}  w/ CMake, IntelliSense, CTest, GDB and GoogleTest functionality on a self-administered \href{https://www.redhat.com/en/technologies/cloud-computing/openshift/container-platform}{OpenShift Container Platform} on zFyre
\item Automated Theia deployment with bash script for installing packages and packaged application into Docker image file with Dockerfile
\item Improved z/CECSIM server development productivity by rewriting traces (15 files and 400 lines) from C filestream API to POSIX system call, to ensure trace visibility when crashing.
\end{rSubsection}

\begin{rSubsection}{MediaTek, Inc.}{--  Hsinchu, Taiwan}{Dec. 2015 - Aug. 2019}{Firmware Engineer, Communication System Design: Developed embedded firmware for 4G and 5G}
\item Developed embedded firmware in MediaTek’s Android platform for 4G LTE and 5G NR physical layer (PHY) digital signal processing
\item Contributed to 6k and maintained 20k lines of code in C/C++ for 3 large-scale projects (over 200k lines and 1,000 developers) including 2019 world’s fastest 5G modem  \href{https://www.mediatek.com/blog/heres-5gs-real-speed-live-connection-demo
}{Helio M70} with downlink throughput of 4.7 Gbps
\item Designed mobile modem receiver modules, implemented OFDM signal processing algorithms accordant with  \href{https://www.3gpp.org/specifications}{3GPP Spec}, integrated build and auto testing process for design and system verification with Perl and Python scripts 
\item Resolved urgent system function and performance issues for over 10 MediaTek’s \href{https://www.mediatek.com/products/smartphones}{Helio X and P series smartphone products} with log examination, signal and procedure analysis, solution formulation and implementation, patch verification and release

\item Received 7 MediaTek vAwards (top 10\% monthly team performance) including  improving 4G downlink data rate by 300\% (14$\rightarrow$42 Mbps) with Massive MIMO feature and reducing 5G control channel decoding computation  cycles by 72\% (3700$\rightarrow$1000 cycles)

\end{rSubsection}

\end{rSection}

%----------------------------------------------------------------------------------------
%	TECHNICAL STRENGTHS SECTION
%----------------------------------------------------------------------------------------

\begin{rSection}{\href{https://stackshare.io/kuangyu0801/kuang-yu-li-techstack}{Technical Strengths}}
\begin{tabular}{ @{} >{ \bfseries}l @{ \hspace{1ex}} l}

{Language} & {Java (expert), C{++} (fluent), Pyhton, JavaScript, Shell Script (familiar) {\bf DevOps} Git, Docker, Kubernetes, Travis CI} \\
{Database} & {SQL: MySQL | NoSQL: PostgreSQL {\bf Web} Spring Boot, MVC, JPA/Hibernate, Data REST, Angular, NGINX}  \\
{Cloud} & {AWS: EC2, Elastic Beanstalk, SQS, RDS, S3, ElastiCache | Google Cloud: Kubernetes Engine, Functions, Pub/Sub} \\
\end{tabular}
%\begin{tabular}{ @{} >{\bfseries}l @{\hspace{6ex}} l }
% Computer Languages & Prolog, Haskell, AWK, Erlang, Scheme, ML \\
% Protocols \& APIs & XML, JSON, SOAP, REST \\
% Databases & MySQL, PostgreSQL, Microsoft SQL \\
% Tools & SVN, Vim, Emacs
%\end{tabular}

\end{rSection}

%----------------------------------------------------------------------------------------
%	EDUCATION SECTION
%----------------------------------------------------------------------------------------

\begin{rSection}{Education}
{\bf Universität Stuttgart }{--  Stuttgart, Germany}  \hfill {Oct. 2019 - May. 2022}\\
{\em M.S. in Information Technology}{, GPA: 3.53/4.0, German Grading: 1.7 (Gut) \\ 
{\bf National Chiao Tung University (NCTU) }{--  Hsinchu, Taiwan} \hfill {Sept. 2009 - Oct. 2015}\\
{\em B.S. in Electrical Engineering and Computer Science}{, GPA: 3.9/4.3}
 | 
\em M.S. in Electronics Engineering and Electronics}{, GPA: 4.27/4.3} 

\end{rSection}

%----------------------------------------------------------------------------------------
%	PROJECT
%----------------------------------------------------------------------------------------

\begin{rSection}{PROJECTS}
\begin{rSubsection}{\href{https://github.com/kuangyu0801/software-defined-networking-ws20/tree/main/sdn-assign-04}{Content-Based Router for Publish-Subscribe Service}} { Java, REST API, HTTP}{Distributed System Lab}{}
\item Developed a publishing service and routing controller in a Java module.  The service receives subscription via HTTP (GET, POST, and
DELETE) and controller provides content-based routing for the service to minimize network traffic and reduce application filtering effort.
\item Developed a "Subscriber" Java application which can subscribe to publishing service via REST API and receive UDP  from publisher.

\end{rSubsection}

\begin{rSubsection}{\href{https://github.com/kuangyu0801/software-defined-networking-ws20/tree/main/sdn-assign-03}{Dynamic Routing for Software-Defined Network}} {Java, TCP, Dijkstra's algorithm}{Distributed System Lab}{}
\item Developed a Java module in Floodlight controller with 2 routing modes in OpenFlow network based on Dijkstra's algorithm.
\item Reactive routes with shortest path. Adaptive routes TCP flow with load balancing by dynamic traffic query on IP and TCP port pair.
\item Verified application with Iperf in MiniNet on Linux and achieves 6x bandwidth increase (582kbs vs 3478kbs) in adaptive mode
\end{rSubsection}

\begin{rSubsection}{\href{https://github.com/kuangyu0801/MobileComputing_SS20_assign04}{Java Application for Wireless Ad-hoc Network}} {Java, UDP, Raspberry Pi}{Mobile Computing}{}
\item Developed 4 Java server and client applications for Flooding and Dynamic Source Routing (DSR) protocols. Flooding achieves high robustness with UDP messages broadcast. DSR achieves reduced data transfer overhead with route discovery in control messages. 
\item Verified on Raspberry Pi in \href {https://www.ipvs.uni-stuttgart.de/}{IPVS}'s mesh 802.11 WiFi network; Applications use \href {https://docs.oracle.com/javase/7/docs/api/java/net/DatagramSocket.html}{java.net.DatagramSocket} for UDP transmission.

\end{rSubsection}
\begin{rSubsection}{\href{https://github.com/kuangyu0801/MobileComputing_SS20_assign03}{Android App for City Temperature with Google Firebase}}{Java, Android}{Mobile Computing}{}
\item Developed an Android application, which can update, subscribe, and calculate daily average of designated city temperature
\item Implemented functions for accessing and querying  data in JSON in a shared Realtime NoSQL database  with Google Firebase API
\end{rSubsection}
\begin{rSubsection}{\href{https://github.com/kuangyu0801/WS19_ComplexNetworkSystem}{ATP Tennis Player Network Analysis}}{Python, Graph, NetworkX}{Complex Network System}{}
\item Developed Python programs to generate complex network and derive structural insights such as Page Rank, Connectivity, Clustering, etc.
\item Implemented algorithms with NetworkX package and built an undirected graph by processing real tennis match statics in csv format.
\item Discovered, visualized, rendered and exported network topology with open-source software Gephi
\end{rSubsection}

\end{rSection}

%%
%\begin{rSection}{HONOR}
%{\bf NCTU Excellent Exchange Student Scholarship}  \hfill {Aug. 2012}\\
%{9,000 USD for exchange to Katholieke Universiteit Leuven, Belgium (top 2\% in NCTU)
%\end{rSection}

%%
%----------------------------------------------------------------------------------------

%----------------------------------------------------------------------------------------
%	EXAMPLE SECTION
%----------------------------------------------------------------------------------------

%\begin{rSection}{Section Name}
% {\bf #Title} \hfill {\em #Time} \\ 
% #Description

%Section content\ldots

%\end{rSection}

%----------------------------------------------------------------------------------------

\end{document}
